% This must be in the first 5 lines to tell arXiv to use pdfLaTeX, which is strongly recommended.
\pdfoutput=1
% In particular, the hyperref package requires pdfLaTeX in order to break URLs across lines.

\documentclass[11pt]{article}

% Remove the "review" option to generate the final version.
\usepackage{random}

% Standard package includes
\usepackage{times}
\usepackage{latexsym}

%\usepackage[utf8]{inputenc}
%\usepackage[T1]{fontenc}
\usepackage{fourier, erewhon}
\usepackage{geometry}
\usepackage{array, caption, floatrow, tabularx, makecell, booktabs}%
\captionsetup{labelfont = sc}
\setcellgapes{3pt}

\usepackage{diagbox}

% For proper rendering and hyphenation of words containing Latin characters (including in bib files)
\usepackage[T1]{fontenc}
% For Vietnamese characters
% \usepackage[T5]{fontenc}
% See https://www.latex-project.org/help/documentation/encguide.pdf for other character sets

% This assumes your files are encoded as UTF8
\usepackage[utf8]{inputenc}
\usepackage{comment}

% This is not strictly necessary, and may be commented out,
% but it will improve the layout of the manuscript,
% and will typically save some space.
\usepackage{microtype}

% If the title and author information does not fit in the area allocated, uncomment the following
%
%\setlength\titlebox{<dim>}
%
% and set <dim> to something 5cm or larger.


% references
\newcommand{\tabref}[1]{\hyperref[tab:#1]{Table~\ref*{tab:#1}}}
\newcommand{\figref}[1]{\hyperref[fig:#1]{Figure~\ref*{fig:#1}}}
\newcommand{\secref}[1]{\hyperref[sec:#1]{Section~\ref*{sec:#1}}}
\newcommand{\defref}[1]{\hyperref[def:#1]{Definition~\ref*{def:#1}}}
\newcommand{\appref}[1]{\hyperref[app:#1]{Appendix~\ref*{app:#1}}}
\newcommand{\chref}[1]{\hyperref[ch:#1]{Chapter~\ref*{ch:#1}}}
\newcommand{\thmref}[1]{\hyperref[thm:#1]{Theorem~\ref*{thm:#1}}}
\newcommand{\lemref}[1]{\hyperref[lem:#1]{Lemma~\ref*{lem:#1}}}
\newcommand{\exref}[1]{\hyperref[ex:#1]{Example~\ref*{ex:#1}}}


\newcommand{\pfour}{P4}
\newcommand{\pc}{p4c}
\newcommand{\petra}{Petr4}
\newcommand{\stf}{stf}
\newcommand{\pversion}{\pfour \textsubscript {16}}

\title{\LARGE Formalizing and Extending \pfour's Type System}

\begin{document}

\author{Parisa Ataei, Ryan Doenges, and Nate Foster\\
  Cornell University \\
  \texttt{\{psa43,rhd89,jnf27\}@cornell.edu}}

\maketitle

The \pfour language currently lacks a precise specification. This
shortcoming is highlighted every time an issue is created that
identifies an ambiguity or incompleteness in the (informal) language
specification\footnote{For
example:
\url{https://github.com/p4lang/p4-spec/issues/876},
\url{https://github.com/p4lang/p4-spec/issues/1221}, and
\url{https://github.com/p4lang/p4-spec/issues/953}.}, or a
discrepancy between the language specification and the reference
compiler\footnote{For example:
\url{https://github.com/p4lang/p4c/issues/3730}.}.

The lack of a precise specification also creates friction whenever
the \pfour Language Design Working Group (LDWG) considers a proposal
for extending the language with new features. In particular, the group
works hard to understand the interactions between new features and
existing features. But without a full formalization, it can be hard to
understand the interaction of the new feature and existing language
features.\footnote{For
example: \url{https://github.com/p4lang/p4-spec/issues/1001}.}

The programming languages community has developed a rich collection of
powerful tools for modeling the syntax and semantics of programming
language. The ``gold standard'' is to use techniques like formal
grammars and inductively defined relations to model the operational
semantics and type systems. But, to date, these tools have not been
applied to \pfour. The Petr4~\cite{petr4} and P4Cub~\cite{p4cub}
efforts are similar, but they focus on core calculi rather than the
full-blown surface syntax.

To address these shortcomings, we have been developing a formalization
of the \pfour syntax, type system, and operational semantics, building
on the Petr4 framework~\cite{petr4}. However, unlike Petr4, we model
the entire surface-level language. The contributions of this work
include:
\begin{itemize}
\item The development of a new front-end for \pfour including
a surface-level intermediate representation (IR) that,
unlike \texttt{p4c}'s IR, directly follows the official grammar
published in the lanugage specification.
\item A type system that models compile-time evaluation, cast insertion,
coercions, type inference, and type checking. Unlike the implementation
in \texttt{p4c}, which uses powerful Hindly-Milner type inference, our
type system is based on a simpler scheme known as local type inference.
\item To illustrate the benefits of our approach, we also present a
proposal for extending \pfour with type families, and we demonstrate
how this extension can be be formalized in our model.
\end{itemize}

We envision several appliations of our work. Language designers can
use it to explore the design of new features. Compiler writers can use
it for differential testing~\cite{gauntlet}. Tool developers can use
it to build linters, static analyzers, language servers, etc. In
our \pfour workshop talk, we will interleave discussion of our
formalization with practical examples and applications.

{\footnotesize
\bibliography{submission}
\bibliographystyle{plain}
}
%\appendix
%
%\section{Example Appendix}
%\label{sec:appendix}
%
%This is an appendix.

\end{document}
